%%
% Copyright (c) 2017 - 2019, Pascal Wagler;  
% Copyright (c) 2014 - 2019, John MacFarlane
% 
% All rights reserved.
% 
% Redistribution and use in source and binary forms, with or without 
% modification, are permitted provided that the following conditions 
% are met:
% 
% - Redistributions of source code must retain the above copyright 
% notice, this list of conditions and the following disclaimer.
% 
% - Redistributions in binary form must reproduce the above copyright 
% notice, this list of conditions and the following disclaimer in the 
% documentation and/or other materials provided with the distribution.
% 
% - Neither the name of John MacFarlane nor the names of other 
% contributors may be used to endorse or promote products derived 
% from this software without specific prior written permission.
% 
% THIS SOFTWARE IS PROVIDED BY THE COPYRIGHT HOLDERS AND CONTRIBUTORS 
% "AS IS" AND ANY EXPRESS OR IMPLIED WARRANTIES, INCLUDING, BUT NOT 
% LIMITED TO, THE IMPLIED WARRANTIES OF MERCHANTABILITY AND FITNESS 
% FOR A PARTICULAR PURPOSE ARE DISCLAIMED. IN NO EVENT SHALL THE 
% COPYRIGHT OWNER OR CONTRIBUTORS BE LIABLE FOR ANY DIRECT, INDIRECT, 
% INCIDENTAL, SPECIAL, EXEMPLARY, OR CONSEQUENTIAL DAMAGES (INCLUDING,
% BUT NOT LIMITED TO, PROCUREMENT OF SUBSTITUTE GOODS OR SERVICES; 
% LOSS OF USE, DATA, OR PROFITS; OR BUSINESS INTERRUPTION) HOWEVER 
% CAUSED AND ON ANY THEORY OF LIABILITY, WHETHER IN CONTRACT, STRICT 
% LIABILITY, OR TORT (INCLUDING NEGLIGENCE OR OTHERWISE) ARISING IN 
% ANY WAY OUT OF THE USE OF THIS SOFTWARE, EVEN IF ADVISED OF THE 
% POSSIBILITY OF SUCH DAMAGE.
%%

%%
% This is the Eisvogel pandoc LaTeX template.
%
% For usage information and examples visit the official GitHub page:
% https://github.com/Wandmalfarbe/pandoc-latex-template
%%

% Options for packages loaded elsewhere
\PassOptionsToPackage{unicode}{hyperref}
\PassOptionsToPackage{hyphens}{url}
\PassOptionsToPackage{dvipsnames,svgnames*,x11names*,table}{xcolor}
%
\documentclass[
  english,
  a4paper,
,tablecaptionabove
]{scrartcl}
\usepackage{lmodern}
\usepackage{setspace}
\setstretch{1.2}
\usepackage{amssymb,amsmath}
\usepackage{ifxetex,ifluatex}
\ifnum 0\ifxetex 1\fi\ifluatex 1\fi=0 % if pdftex
  \usepackage[T1]{fontenc}
  \usepackage[utf8]{inputenc}
  \usepackage{textcomp} % provide euro and other symbols
\else % if luatex or xetex
  \usepackage{unicode-math}
  \defaultfontfeatures{Scale=MatchLowercase}
  \defaultfontfeatures[\rmfamily]{Ligatures=TeX,Scale=1}
\fi
% Use upquote if available, for straight quotes in verbatim environments
\IfFileExists{upquote.sty}{\usepackage{upquote}}{}
\IfFileExists{microtype.sty}{% use microtype if available
  \usepackage[]{microtype}
  \UseMicrotypeSet[protrusion]{basicmath} % disable protrusion for tt fonts
}{}
\makeatletter
\@ifundefined{KOMAClassName}{% if non-KOMA class
  \IfFileExists{parskip.sty}{%
    \usepackage{parskip}
  }{% else
    \setlength{\parindent}{0pt}
    \setlength{\parskip}{6pt plus 2pt minus 1pt}}
}{% if KOMA class
  \KOMAoptions{parskip=half}}
\makeatother
\usepackage{xcolor}
\definecolor{default-linkcolor}{HTML}{A50000}
\definecolor{default-filecolor}{HTML}{A50000}
\definecolor{default-citecolor}{HTML}{4077C0}
\definecolor{default-urlcolor}{HTML}{4077C0}
\IfFileExists{xurl.sty}{\usepackage{xurl}}{} % add URL line breaks if available
\IfFileExists{bookmark.sty}{\usepackage{bookmark}}{\usepackage{hyperref}}
\hypersetup{
  pdftitle={PPM Project Report},
  pdfauthor={Callum Axon (N0727303); Callum Carney (N0741707); Jordan Brightmore (N0732961); Finlay McKinnon(N0743587); Vital Harachka (N0731739); Wing Chiang (T0086366)},
  colorlinks=true,
  linkcolor=darkgray,
  filecolor=default-filecolor,
  citecolor=default-citecolor,
  urlcolor=default-urlcolor,
  breaklinks=true,
  pdfcreator={LaTeX via pandoc with the Eisvogel template}}
\urlstyle{same} % disable monospaced font for URLs
\usepackage[margin=2.5cm,includehead=true,includefoot=true,centering]{geometry}
\usepackage[export]{adjustbox}
\usepackage{graphicx}
\setlength{\emergencystretch}{3em}  % prevent overfull lines
\providecommand{\tightlist}{%
  \setlength{\itemsep}{0pt}\setlength{\parskip}{0pt}}
\setcounter{secnumdepth}{-\maxdimen} % remove section numbering

% Make use of float-package and set default placement for figures to H
\usepackage{float}
\floatplacement{figure}{H}

\usepackage{pdflscape}
\ifxetex
    % See issue https://github.com/reutenauer/polyglossia/issues/127
  \renewcommand*\familydefault{\sfdefault}
  % Load polyglossia as late as possible: uses bidi with RTL langages (e.g. Hebrew, Arabic)
  \usepackage{polyglossia}
  \setmainlanguage[]{english}
\else
  \usepackage[shorthands=off,main=english]{babel}
\fi

\title{PPM Project Report}
\usepackage{etoolbox}
\makeatletter
\providecommand{\subtitle}[1]{% add subtitle to \maketitle
  \apptocmd{\@title}{\par {\large #1 \par}}{}{}
}
\makeatother
\subtitle{Magnus Frater System}
\author{Callum Axon (N0727303) \and Callum Carney (N0741707) \and Jordan Brightmore (N0732961) \and Finlay McKinnon(N0743587) \and Vital Harachka (N0731739) \and Wing Chiang (T0086366)}
\date{}





%%
%% added
%%

%
% language specification
%
% If no language is specified, use English as the default main document language.
%


%
% for the background color of the title page
%
\usepackage{pagecolor}
\usepackage{afterpage}

%
% break urls
%
\PassOptionsToPackage{hyphens}{url}

%
% When using babel or polyglossia with biblatex, loading csquotes is recommended 
% to ensure that quoted texts are typeset according to the rules of your main language.
%
\usepackage{csquotes}

%
% captions
%
\definecolor{caption-color}{HTML}{777777}
\usepackage[font={stretch=1.2}, textfont={color=caption-color}, position=top, skip=4mm, labelfont=bf, singlelinecheck=false, justification=raggedright]{caption}
\setcapindent{0em}

%
% blockquote
%
\definecolor{blockquote-border}{RGB}{221,221,221}
\definecolor{blockquote-text}{RGB}{119,119,119}
\usepackage{mdframed}
\newmdenv[rightline=false,bottomline=false,topline=false,linewidth=3pt,linecolor=blockquote-border,skipabove=\parskip]{customblockquote}
\renewenvironment{quote}{\begin{customblockquote}\list{}{\rightmargin=0em\leftmargin=0em}%
\item\relax\color{blockquote-text}\ignorespaces}{\unskip\unskip\endlist\end{customblockquote}}

%
% Source Sans Pro as the de­fault font fam­ily
% Source Code Pro for monospace text
%
% 'default' option sets the default 
% font family to Source Sans Pro, not \sfdefault.
%
\usepackage[default]{sourcesanspro}
\usepackage{sourcecodepro}

% XeLaTeX specific adjustments for straight quotes: https://tex.stackexchange.com/a/354887
% This issue is already fixed (see https://github.com/silkeh/latex-sourcecodepro/pull/5) but the 
% fix is still unreleased.
% TODO: Remove this workaround when the new version of sourcecodepro is released on CTAN.
\ifxetex
\makeatletter
\defaultfontfeatures[\ttfamily]
  { Numbers   = \sourcecodepro@figurestyle,
    Scale     = \SourceCodePro@scale,
    Extension = .otf }
\setmonofont
  [ UprightFont    = *-\sourcecodepro@regstyle,
    ItalicFont     = *-\sourcecodepro@regstyle It,
    BoldFont       = *-\sourcecodepro@boldstyle,
    BoldItalicFont = *-\sourcecodepro@boldstyle It ]
  {SourceCodePro}
\makeatother
\fi

%
% heading color
%
\definecolor{heading-color}{RGB}{40,40,40}
\addtokomafont{section}{\color{heading-color}}
% When using the classes report, scrreprt, book, 
% scrbook or memoir, uncomment the following line.
%\addtokomafont{chapter}{\color{heading-color}}

%
% variables for title and author
%
\usepackage{titling}
\title{PPM Project Report}
\author{Callum Axon (N0727303), Callum Carney (N0741707), Jordan Brightmore (N0732961), Finlay McKinnon(N0743587), Vital Harachka (N0731739), Wing Chiang (T0086366)}

%
% tables
%

%
% remove paragraph indention
%
\setlength{\parindent}{0pt}
\setlength{\parskip}{6pt plus 2pt minus 1pt}
\setlength{\emergencystretch}{3em}  % prevent overfull lines

%
%
% Listings
%
%


%
% header and footer
%
\usepackage{fancyhdr}

\fancypagestyle{eisvogel-header-footer}{
  \fancyhead{}
  \fancyfoot{}
  \lhead[]{PPM Project Report}
  \chead[]{}
  \rhead[PPM Project Report]{}
  \lfoot[\thepage]{Callum Axon (N0727303), Callum Carney (N0741707), Jordan Brightmore (N0732961), Finlay McKinnon(N0743587), Vital Harachka (N0731739), Wing Chiang (T0086366)}
  \cfoot[]{}
  \rfoot[Callum Axon (N0727303), Callum Carney (N0741707), Jordan Brightmore (N0732961), Finlay McKinnon(N0743587), Vital Harachka (N0731739), Wing Chiang (T0086366)]{\thepage}
  \renewcommand{\headrulewidth}{0.4pt}
  \renewcommand{\footrulewidth}{0.4pt}
}
\pagestyle{eisvogel-header-footer}

%%
%% end added
%%

\begin{document}

%%
%% begin titlepage
%%

\begin{titlepage}
\newgeometry{left=6cm}
\definecolor{titlepage-color}{HTML}{06386e}
\newpagecolor{titlepage-color}\afterpage{\restorepagecolor}
\newcommand{\colorRule}[3][black]{\textcolor[HTML]{#1}{\rule{#2}{#3}}}
\begin{flushleft}
\noindent
\\[-1em]
\color[HTML]{FFFFFF}
\makebox[0pt][l]{\colorRule[FFFFFF]{1.3\textwidth}{1pt}}
\par
\noindent

{ \setstretch{1.4}
\vfill
\noindent {\huge \textbf{\textsf{PPM Project Report}}}
\vskip 1em
{\Large \textsf{Magnus Frater System}}
\vskip 2em
\noindent
{\Large \textsf{Callum Axon (N0727303), Callum Carney (N0741707), Jordan Brightmore (N0732961), Finlay McKinnon(N0743587), Vital Harachka (N0731739), Wing Chiang (T0086366)}
\vfill
}

\noindent
\includegraphics[width=60pt, left]{./images/ntu-logo.png}

\textsf{}}
\end{flushleft}
\end{titlepage}
\restoregeometry

%%
%% end titlepage
%%



\hypertarget{abstract}{%
\section{Abstract}\label{abstract}}

Magnus Frater (or Big Brother) has been created to help tackle the
ongoing issue of security within large open campuses and premises, these
sorts of locations inherently have an increased potential for intrusion
through unmonitored sections of land. The group analysed the recent
spree of attacks on schools and offices (for example the shooting that
occurred at the YouTube headquarters in 2018) and found that in a large
amount of these attacks there were open doors and spaces that allowed
the attacker to enter with ease. As a consequence to this, the idea of
creating a facial recognition system to analyse and report known and
unknown people within a campus/large open setting was conceived.

As mentioned, the main purpose of the project was to create a system
that would accurately detect and report people walking around an area to
the associated security team, this data would differentiate between
employees or authorised users and unknown people by linking into the
companies employee/student database. Not only would this allow a
security team to monitor who is within a set area at any one time, but
it would also allow administrative users to track any persons movements
and activities within a set time frame, through tracking of the targets
face across multiple cameras. Another advantage to this project is that
administrative users can view analytics in relation to the usage of
campus properties, an example use case for this would be within a
University. Admins could check what buildings within the campus are
being utilised most by students.

After the main purpose behind the project was defined, the group decided
on how to proceed in regards to the requirements for the project, most
importantly how we should proceed with splitting up the individual
hardware and software components so that the system could functions
within any scenario or environment. It was decided that there will be 4
different modules, these being:

\begin{enumerate}
\def\labelenumi{\arabic{enumi}.}
\tightlist
\item
  A Raspberry Pi that would be responsible for processing any facial
  data that is captured by the camera
\item
  A Camera module that would connect directly to the Raspberry Pi and
  provide images to the Raspberry Pi
\item
  A website created for administrators and security personnel to
  administer and manage hits/rejections.
\item
  An API (Application Programming Interface) used within the website and
  the Raspberry Pi for collation and provision of data.
\end{enumerate}

These modules will work together to create the Cameras that report
facial data and the web interface that is used to manage the data
received by the camera, the connection between these modules was
outlined in the design documentation (for example the Data Flow Diagram
and Entity Relationship Diagram).

Once the components and requirements were completed, the group began to
consider which programming languages and setups would be best suited for
the type of project this is (Facial Recognition with Web Related
components). It was clear that Python should be used for the facial
recognition section of the project due to its strong existing libraries.
NodeJS would be used for the Web Frontend, PHP would be used to power
the backend API that links all of the components together and the API
would be using a MySQL database to hold all of the data. The system
would work in the following way:

\begin{enumerate}
\def\labelenumi{\arabic{enumi}.}
\tightlist
\item
  The Camera feeds data to the Raspberry Pi
\item
  The Python application on the Raspberry Pi calculates if a face is
  present
\item
  Any potential face found is sent to the API where corresponding facial
  data is requested from the database
\item
  If no corresponding data is found, then the face is unknown, otherwise
  the image will be linked to the person the face associates with.
\item
  The Website will update using data from the API to show new
  detections, known or unknown.
\end{enumerate}

Once the product had been developed, testing took place to ensure that
the facial recognition software worked from a variety of different
distances and in unfavourable circumstances (heavy rain, fog, etc).
While some of the tests passed, others failed to detect faces when they
were present, however this only occurred in extreme circumstances. We
made small enhancements to the facial detection algorithm to improve its
effectiveness during these scenarios.

Due to the nature of this system, there are a lot of potential legal and
ethical issues, people may not consent to the recording of their faces,
people may not wish to have their faces processed and stored by this
system. Therefore it was important for us to implement a blacklist
system that would stop the system from performing facial data
processing, however, this is a complex system because we first need to
process a persons face to understand what to blacklist, which could
cause further legal or ethical issues.

\hypertarget{table-of-contents}{%
\section{Table of Contents}\label{table-of-contents}}

\hypertarget{references}{%
\section{References}\label{references}}

{[}{]} {[}@infotech\_2017{]} {[}@wayner\_2019{]}

\end{document}
